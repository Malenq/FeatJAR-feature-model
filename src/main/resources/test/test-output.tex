\documentclass[border=5pt]{standalone}
%---required packages & variable definitions------------------------------------
\usepackage{forest}
\usepackage{amsmath}
\usepackage{xcolor}
\usetikzlibrary{angles}
\usetikzlibrary{positioning}
\definecolor{drawColor}{RGB}{128 128 128}
\newcommand{\circleSize}{0.25em}
%-------------------------------------------------------------------------------
%---Define the style of the tree------------------------------------------------
\forestset{
    /tikz/mandatory/.style={
        circle,fill=drawColor,
        draw=drawColor,
        inner sep=\circleSize
    },
    /tikz/optional/.style={
        circle,
        fill=white,
        draw=drawColor,
        inner sep=\circleSize
    },
    featureDiagram/.style={
        for tree={
            draw = drawColor,
            edge = {draw=drawColor},
            anchor=north,
            parent anchor = south,
            child anchor = north,
            l sep = 2em,
            s sep = 1em,
        }
    },
    /tikz/abstract/.style={
        fill = blue!85!cyan!5,
        draw = drawColor
    },
    /tikz/concrete/.style={
        fill = blue!85!cyan!20,
        draw = drawColor
    },
    mandatory/.style={
        edge label+={
            node [mandatory] {}
        }
    },
    optional/.style={
        edge label+={
            node [optional] {}
        }
    },
    featurecardinality/.style n args={2}{
        edge label+={
            node[midway,fill=white,font=\scriptsize]{#1,#2}
        }
    },
    or/.style n args={3}{
        tikz+={
            \path
            (!{current, n=#1}.parent) coordinate (A) -- (!c.children) coordinate (B) -- (!{current, n=#2}.parent) coordinate (C) -- (!{current, n=#3}.parent) coordinate (D)
            let \p1 = (A), \p2 = (B), \p3 = (C), \p4 = (D), \n1 = {veclen(\x2 - \x1, \y2 - \y1)}, \n2 = {veclen(\x2 - \x3, \y2 - \y3)}, \n3 = {veclen(\x2 - \x4, \y2 - \y4)}
            in pic[
            fill=drawColor,
            angle radius={min(\n1/2,\n2/2,\n3/2)}
            ]{angle};
        }
    },
    alternative/.style n args={3}{
        tikz+={
            \path
            (!{current, n=#1}.parent) coordinate (A) -- (!c.children) coordinate (B) -- (!{current, n=#2}.parent) coordinate (C) -- (!{current, n=#3}.parent) coordinate (D)
            let \p1 = (A), \p2 = (B), \p3 = (C), \p4 = (D), \n1 = {veclen(\x2 - \x1, \y2 - \y1)}, \n2 = {veclen(\x2 - \x3, \y2 - \y3)}, \n3 = {veclen(\x2 - \x4, \y2 - \y4)}
            in pic[
            draw=drawColor,
            angle radius={min(\n1/2,\n2/2,\n3/2)}
            ]{angle};
        }
    },
    groupcardinality/.style n args={5}{
        tikz+={
            \path (!{current, n=#1}.parent) coordinate (A) -- (!c.children) coordinate (B) -- (!{current, n=#2}.parent) coordinate (C) -- (!{current, n=#3}.parent) coordinate (D)
            let \p1 = (A), \p2 = (B), \p3 = (C), \p4 = (D), \n1 = {veclen(\x2 - \x1, \y2 - \y1)}, \n2 = {veclen(\x2 - \x3, \y2 - \y3)}, \n3 = {veclen(\x2 - \x4, \y2 - \y4)}
            in pic[
            draw=drawColor,
            angle radius={min(\n1/2,\n2/2,\n3/2)},
            pic text={#4,#5},
            pic text options={
                scale=0.6,
                fill=white,
                inner sep=0.3pt
            }
            ]{angle};
        }
    },
    /tikz/placeholder/.style={
    },
    collapsed/.style={
        rounded corners,
        no edge,
        for tree={
            fill opacity=0,
            draw opacity=0,
            l = 0em,
        }
    },
    /tikz/hiddenNodes/.style={
        midway,
        rounded corners,
        draw=drawColor,
        fill=white,
        minimum size = 1.2em,
        minimum width = 0.8em,
        scale=0.9
    }
}
%-------------------------------------------------------------------------------
\begin{document}
	%---The Feature Diagram-----------------------------------------------------
\begin{forest}
	featureDiagram
	[\multicolumn{2}{c}{Hello} \\\hline
,align=ll,abstract[\multicolumn{2}{c}{Feature} \\\hline
,align=ll,abstract,featurecardinality={0}{2},alternative={1}{2}{1},or={3}{4}{3},groupcardinality={5}{6}{5}{7}{8}[\multicolumn{2}{c}{Wonderful1} \\\hline
\small\texttt{who (String)} &\small\texttt{= you} \\
\small\texttt{when (String)} &\small\texttt{= now} \\
,align=ll,concrete][\multicolumn{2}{c}{Beautiful1} \\\hline
,align=ll,concrete][\multicolumn{2}{c}{Wonderful2} \\\hline
,align=ll,concrete][\multicolumn{2}{c}{Beautiful2} \\\hline
,align=ll,concrete,groupcardinality={1}{3}{2}{0}{2}[\multicolumn{2}{c}{Meaningful1} \\\hline
,align=ll,concrete][\multicolumn{2}{c}{Meaningful2} \\\hline
,align=ll,concrete][\multicolumn{2}{c}{Meaningful3} \\\hline
,align=ll,concrete]][\multicolumn{2}{c}{Wonderful3} \\\hline
\small\texttt{who (String)} &\small\texttt{= you} \\
,align=ll,concrete][\multicolumn{2}{c}{Beautiful3} \\\hline
,align=ll,concrete]][\multicolumn{2}{c}{World1} \\\hline
\small\texttt{size (Double)} &\small\texttt{= 6000.0} \\
\small\texttt{population (Integer)} &\small\texttt{= 1} \\
,align=ll,concrete,optional][\multicolumn{2}{c}{World2} \\\hline
,align=ll,concrete,mandatory]]	
	\matrix [anchor=north west] at (current bounding box.north east) {
		\node [placeholder] {}; \\
	};
	\matrix [draw=drawColor,anchor=north west] at (current bounding box.north east) {
		\node [label=center:\underline{Legend:}] {}; \\
		\node [abstract,label=right:Abstract Feature] {}; \\
		\node [concrete,label=right:Concrete Feature] {}; \\
		\node [mandatory,label=right:Mandatory] {}; \\
		\node [optional,label=right:Optional] {}; \\
 \filldraw[drawColor] (0.1,0) - +(-0,-0.2) - +(0.2,-0.2)- +(0.1,0); 
			\draw[drawColor] (0.1,0) -- +(-0.2, -0.4);
			\draw[drawColor] (0.1,0) -- +(0.2,-0.4);
			\fill[drawColor] (0,-0.2) arc (240:300:0.2);
		\node [label=right:Or Group] {}; \\
			\draw[drawColor] (0.1,0) -- +(-0.2, -0.4);
			\draw[drawColor] (0.1,0) -- +(0.2,-0.4);
			\draw[drawColor] (0,-0.2) arc (240:300:0.2);
		\node [label=right:Alternative Group] {}; \\
	};
	\matrix [below=1mm of current bounding box] {
	\node {\( \text{World1} \land \text{Wonderful1} \)}; \\
	\node {\( \text{World2} \Rightarrow ( \text{Beautiful2} \Leftrightarrow \text{Beautiful3} )\)}; \\
	};
\end{forest}

	%---------------------------------------------------------------------------
\end{document}